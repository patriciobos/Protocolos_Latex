\documentclass[aspectratio=169]{beamer}
%la opcion hangout es para complilar en modo imprimible
%\documentclass[hangout]{beamer}

\mode<presentation>
{
  \usetheme{Berkeley}
  \setbeamercovered{transparent}
  \setbeamertemplate{navigation symbols}{}
}

\usepackage[spanish]{babel}
\usepackage[utf8]{inputenc}
%\usepackage{times}
\usepackage{tikz}
\usepackage{textpos}

%\usetikzlibrary{shapes,arrows}
\setbeamerfont{author}{size=\large}
\setbeamerfont{institute}{size=\normalsize\bfseries}
\setbeamerfont{title}{size=\Large\bfseries}
\setbeamerfont{subtitle}{size=\huge}

\definecolor{darkblue}{RGB}{51,51,179}

\title[802.15.4 LR-WPAN]{Protocolos de Comunicación en Sistemas Embebidos}
\subtitle{802.15.4 LR-WPAN}
\author[]{Ing. Patricio Bos, Esp. Ing. Juan Montilla}
\institute[LSE-FIUBA]{Laboratorio de Sistemas Embebidos - FIUBA}
\date{01/06/2016}

%\subtitle{Framework para aplicaciones de control de ambientes}
\titlegraphic{\includegraphics[width=5cm]{./imagenes/red.jpg}}


\subject{Protocolos y Comunicaciones: 802.15.4 LR-WPAN. Carrera de Especialización en Sistemas Embebidos}
% This is only inserted into the PDF information catalog. Can be left
% out. 

\pgfdeclareimage[height=1.5cm]{university-logo}{./imagenes/logo-facu-inverso.png}
\logo{\pgfuseimage{university-logo}}


% If you wish to uncover everything in a step-wise fashion, uncomment
% the following command: 

\beamerdefaultoverlayspecification{<+->}
  
\begin{document}

%la magia del begingroup es para que titlepage quede centrada, sin eso queda
%corrida en el ancho del sidebar
\begingroup
\makeatletter
\setlength{\hoffset}{-.5\beamer@sidebarwidth}
\makeatother
\begin{frame}[plain,noframenumbering]
  \titlepage
\end{frame}

\endgroup



\begin{frame}{\textbf{Organización de la presentación}}
  \tableofcontents
  % You might wish to add the option [pausesections]
\end{frame}
%
%
%
\section{Introducción}

\begin{frame}{Grupos de trabajo IEEE} 

\begin{minipage}[c]{1.0\linewidth}
	\begin{minipage}[c]{0.6\linewidth}
		\begin{itemize}
			\item IEEE 802: Desarrollar estándares para redes de área local y metropolitana (LAN y MAN)
			\begin{itemize}
				\item IEEE 802.3: Ethernet
				\item IEEE 802.11: Wi-fi
				\item ...
			\end{itemize}
			\vspace{10px}
			\item IEEE 802.15: Redes inalámbricas de área personal (WPAN)
			\vspace{5px}
			\begin{itemize}
				\item IEEE 802.15.1: Bluetooth
				\item IEEE 802.15.4: WPANs de baja tasa de transferencia de datos (LR-WPAN)
			\end{itemize}
		\end{itemize}
	\end{minipage}
	\begin{minipage}[c]{0.35\linewidth}
		\begin{figure}[H]
			\uncover<2>{\includegraphics[width=1\textwidth]{./imagenes/OSI_Model_v1.pdf}}
			\label{OSI_model}
			\caption{Modelo de capas OSI}
		\end{figure}	  	  	
	\end{minipage}
\end{minipage}
	
\end{frame}

\begin{frame}{IEEE 802.15.4: LR-WPAN}
	\begin{itemize}
		\item Versiones: 802.15.4:2003, 802.15.4:2006 y \textbf{802.15.4:2011}
		\item Define:
		\begin{itemize}
			\item Nivel físico (PHY)
			\item Control de acceso al medio (MAC)
		\end{itemize}
		\item Características:
		\begin{itemize}
			\item Comunicaciones simples de bajo costo. 
			\item Bajas tasas de transferencia (throughput).
			\item Para aplicaciones con limitaciones de potencia.
			\item Confiabilidad en la transferencia de datos.
		\end{itemize}
	\end{itemize}
	
\end{frame}

\section{WPAN}

\begin{frame}{\textbf{Componentes y Topología}}
%\fontsize{14pt}{15}\selectfont  % para agrandar la letra sacar el primer "%"
% El primer minipage es un marco para las otras dos, que parten la pantalla en dos horizontalmente.
% Con 0.6\linewidth le indicás que porcentaje del ancho de la página debe tener la minipage
\begin{minipage}[c]{1.0\linewidth}

	\begin{minipage}[c]{0.6\linewidth}
		\begin{itemize}
			\item Full-function device (FFD): Capaz de ser PAN coordinator o coordinator. 
			\vspace{10px}
			\item Reduced-function device (RFD): Puede ser implementado usando los mínimos recursos y capacidad de memoria.
			\vspace{10px}
	  	\end{itemize}	
	\end{minipage}

	\begin{minipage}[c]{0.35\linewidth}
		\begin{figure}[H]
		%{\includegraphics[width=1\textwidth]{./imagenes/acuario.jpg}}
		\end{figure}	  	  	
	\end{minipage}

\end{minipage}
\end{frame}

\begin{frame}{\textbf{Alternativa: Árbol de Cluster}}
%\fontsize{14pt}{15}\selectfont
\begin{minipage}[c]{1.0\linewidth}
	\begin{minipage}[c]{0.6\linewidth}
		\begin{itemize}
			\item Área de cobertura vs Latencia del mensaje 
			\vspace{10px}
	  	\end{itemize}	
	\end{minipage}
	\begin{minipage}[c]{0.35\linewidth}
		\begin{figure}[H]
			%{\includegraphics[width=1\textwidth]{./imagenes/acuario.jpg}}
		\end{figure}	  	  	
	\end{minipage}
\end{minipage}
\end{frame}


\begin{frame}{\textbf{Arquitectura del estándar}}

%\fontsize{14pt}{15}\selectfont
\begin{minipage}[c]{1.0\linewidth}
	
	\begin{minipage}[c]{0.6\linewidth}
		\begin{itemize}
			\item MAC Sublayer: Acceso al canal físico para todo tipo de transferencias (gestion de beacon y GTS, acceso al canal, validación de frame, ack, asociación y desasociación). 
			\vspace{10px}
			\item Physical Layer (PHY): Mecanismo de control del transceiver RF (Activación y Desactivación de RF, ED, LQI, selección de canal, CCA y transmisión y recepción de paquetes del canal físico). 
			\vspace{10px}
	  	\end{itemize}	
	  \end{minipage}

	  \begin{minipage}[c]{0.35\linewidth}
		\begin{figure}[H]
		%	{\includegraphics[width=1\textwidth]{./imagenes/acuario.jpg}}
		\end{figure}	  	  	
	  \end{minipage}
	\end{minipage}

\end{frame}

\section{Estándar 802.15.4}


\begin{frame}{\textbf{Transferencia de Datos}}
%%\fontsize{14pt}{15}\selectfont
%\begin{minipage}[c]{1.0\linewidth}
%	\begin{minipage}[c]{0.6\linewidth}
%		\begin{itemize}
%			\item Frame: (Dev -> Coord, con beacon)
%			\vspace{10px}
%	  	\end{itemize}	
%	\end{minipage}
%	\begin{minipage}[c]{0.35\linewidth}
%		\begin{figure}[H]
%			\includegraphics[width=1\textwidth]{./imagenes/dev-coord-beacon.jpg}
%		\end{figure}	  	  	
%	\end{minipage}
%\end{minipage}

Frame: (Dev -> Coord, con beacon)
\vspace{10px}

\begin{figure}[H]
			\includegraphics[height=.8\textheight]{./imagenes/dev-coord-beacon.jpg}
		\end{figure}	 
\end{frame}

\begin{frame}{\textbf{Transferencia de Datos}}
%\fontsize{14pt}{15}\selectfont
\begin{minipage}[c]{1.0\linewidth}
	\begin{minipage}[c]{0.6\linewidth}
		\begin{itemize}
			\item Frame: (Dev -> Coord, sin beacon)
			\vspace{10px}
	  	\end{itemize}	
	\end{minipage}
	\begin{minipage}[c]{0.35\linewidth}
		\begin{figure}[H]
			\includegraphics[width=1\textwidth]{./imagenes/dev-coord-sinbeacon.jpg}
		\end{figure}	  	  	
	\end{minipage}
\end{minipage}
\end{frame}

\begin{frame}{\textbf{Transferencia de Datos}}
%\fontsize{14pt}{15}\selectfont
\begin{minipage}[c]{1.0\linewidth}
	\begin{minipage}[c]{0.6\linewidth}
		\begin{itemize}
			\item Frame: (Coord -> Dev, con beacon)
			\vspace{10px}
	  	\end{itemize}	
	\end{minipage}
	\begin{minipage}[c]{0.35\linewidth}
		\begin{figure}[H]
			\includegraphics[width=1\textwidth]{./imagenes/coord-dev-beacon.jpg}
		\end{figure}	  	  	
	\end{minipage}
\end{minipage}
\end{frame}

\begin{frame}{\textbf{Transferencia de Datos}}
%\fontsize{14pt}{15}\selectfont
\begin{minipage}[c]{1.0\linewidth}
	\begin{minipage}[c]{0.6\linewidth}
		\begin{itemize}
			\item Frame: (Coord -> Dev, sin beacon)
			\vspace{10px}
	  	\end{itemize}	
	\end{minipage}
	\begin{minipage}[c]{0.35\linewidth}
		\begin{figure}[H]
			\includegraphics[width=1\textwidth]{./imagenes/coord-dev-sinbeacon.jpg}
		\end{figure}	  	  	
	\end{minipage}
\end{minipage}
\end{frame}

\begin{frame}{\textbf{Transferencia de Datos}}
%\fontsize{14pt}{15}\selectfont
\begin{minipage}[c]{1.0\linewidth}
	\begin{minipage}[c]{0.6\linewidth}
		\begin{itemize}
			\item Frame: Beancon
			\vspace{10px}
	  	\end{itemize}	
	\end{minipage}
	\begin{minipage}[c]{0.35\linewidth}
		\begin{figure}[H]
			\includegraphics[width=1\textwidth]{./imagenes/beacon.jpg}
		\end{figure}	  	  	
	\end{minipage}
\end{minipage}
\end{frame}

\begin{frame}{\textbf{Transferencia de Datos}}
%\fontsize{14pt}{15}\selectfont
\begin{minipage}[c]{1.0\linewidth}
	\begin{minipage}[c]{0.6\linewidth}
		\begin{itemize}
			\item Frame: Datos
			\vspace{10px}
	  	\end{itemize}	
	\end{minipage}
	\begin{minipage}[c]{0.35\linewidth}
		\begin{figure}[H]
			\includegraphics[width=1\textwidth]{./imagenes/data.jpg}
		\end{figure}	  	  	
	\end{minipage}
\end{minipage}
\end{frame}

\begin{frame}{\textbf{Transferencia de Datos}}
%\fontsize{14pt}{15}\selectfont
\begin{minipage}[c]{1.0\linewidth}
	\begin{minipage}[c]{0.6\linewidth}
		\begin{itemize}
			\item Frame: Ack
			\vspace{10px}
	  	\end{itemize}	
	\end{minipage}
	\begin{minipage}[c]{0.35\linewidth}
		\begin{figure}[H]
			\includegraphics[width=1\textwidth]{./imagenes/ack.jpg}
		\end{figure}	  	  	
	\end{minipage}
\end{minipage}
\end{frame}

\begin{frame}{\textbf{Transferencia de Datos}}
%\fontsize{14pt}{15}\selectfont
\begin{minipage}[c]{1.0\linewidth}
	\begin{minipage}[c]{0.6\linewidth}
		\begin{itemize}
			\item Frame: MAC Command
			\vspace{10px}
	  	\end{itemize}	
	\end{minipage}
	\begin{minipage}[c]{0.35\linewidth}
		\begin{figure}[H]
			\includegraphics[width=1\textwidth]{./imagenes/maccommand.jpg}
		\end{figure}	  	  	
	\end{minipage}
\end{minipage}
\end{frame}

\begin{frame}{\textbf{Transferencia de Datos}}
%\fontsize{14pt}{15}\selectfont
\begin{minipage}[c]{1.0\linewidth}
	\begin{minipage}[c]{0.6\linewidth}
		\begin{itemize}
			\item Frame Control Field
			\vspace{10px}
	  	\end{itemize}	
	\end{minipage}
	\begin{minipage}[c]{0.35\linewidth}
		\begin{figure}[H]
			\includegraphics[width=1\textwidth]{./imagenes/FCF.jpg}
		\end{figure}
		\begin{figure}[H]
			\includegraphics[width=1\textwidth]{./imagenes/addressingmode.jpg}
		\end{figure}	
		\begin{figure}[H]
			\includegraphics[width=1\textwidth]{./imagenes/frametype.jpg}
		\end{figure}		  	  	
	\end{minipage}
\end{minipage}
\end{frame}

\begin{frame}{\textbf{Control de Acceso al Medio}}
%\fontsize{14pt}{15}\selectfont
\begin{minipage}[c]{1.0\linewidth}
	\begin{minipage}[c]{0.6\linewidth}
		\begin{itemize}
			\item Slotted CSMA/CA vs Unslotted CSMA/CA
			\vspace{10px}
	  	\end{itemize}	
	\end{minipage}
	\begin{minipage}[c]{0.35\linewidth}
		\begin{figure}[H]
			\includegraphics[width=1\textwidth]{./imagenes/superframe.jpg}
		\end{figure}
		\begin{figure}[H]
			\includegraphics[width=1\textwidth]{./imagenes/unsolotted.jpg}
		\end{figure}		  	  	
	\end{minipage}
\end{minipage}
\end{frame}

\begin{frame}{\textbf{Modulación}}
%\fontsize{14pt}{15}\selectfont
\begin{minipage}[c]{1.0\linewidth}
\begin{figure}[H]
	\includegraphics[height=1\textheight]{./imagenes/modulaciones.jpg}
		\end{figure}	
\end{minipage}
\end{frame}

\section{TI2520}

\begin{frame}{\textbf{Circuito de Aplicación Típico}}
	
	\begin{figure}[H]
		\includegraphics[height=1\textheight]{./imagenes/applicationcircuit.jpg}
	\end{figure}	

\end{frame}

\begin{frame}{\textbf{Diagrama Funcional}}

	\begin{figure}[H]
		\includegraphics[height=1\textheight]{./imagenes/diagrama.jpg}
	\end{figure}	

\end{frame}

\begin{frame}{\textbf{Tx FIFO}}

	\begin{figure}[H]
		\includegraphics[height=1\textheight]{./imagenes/txfifo.jpg}
	\end{figure}	

\end{frame}

\begin{frame}{\textbf{Filtering Algorithm}}

	\begin{figure}[H]
		\includegraphics[height=1\textheight]{./imagenes/filtering.jpg}
	\end{figure}	

\end{frame}


\begin{frame}{\textbf{Matching Algorithm}}

	\begin{figure}[H]
		\includegraphics[height=1\textheight]{./imagenes/matching.jpg}
	\end{figure}	

\end{frame}


\section{Mote}

%\section[Problema]{Planteo del problema a resolver}
%
%\begin{frame}{\textbf{Planteo del problema a resolver}}
%\fontsize{18pt}{15}\selectfont
%	\begin{itemize}
%		\item {¿Qué hace falta medir?}
%		\vspace{20px}
%		\item ¿Qué hace falta controlar?
%		\vspace{20px}
%		\item ¿Sobre qué hace falta alertar?
%		\vspace{10px}
%	\end{itemize}
%\end{frame}
%
%\begin{frame}{\textbf{¿Qué hace falta medir? - Sensores}}
%\fontsize{14pt}{15}\selectfont
%
%\begin{minipage}[c]{1.0\linewidth}
%\begin{minipage}[c]{0.6\linewidth}
%      \centering
%      \begin{itemize}
%      	\item Temperatura
%		\vspace{10px}
%		\item pH
%		\vspace{10px}
%		\item Nivel de agua
%		\vspace{10px}
%		\item otros
%			\begin{itemize}
%				\item Conductividad
%				\item Nitratos
%				\item etc...
%			\end{itemize}
%	\end{itemize}
% \end{minipage}
%  \begin{minipage}[c]{0.35\linewidth}
%	\begin{figure}[H]
%	%	{\includegraphics[width=1\textwidth]{./imagenes/sensor_temp}\vspace{5px}}
%	%	{\includegraphics[width=1\textwidth]{./imagenes/sensor_nivel}}	
%	\end{figure}	  	  	
%  \end{minipage}
%\end{minipage}
%\end{frame}
%	
%
%\begin{frame}{\textbf{¿Qué hace falta controlar? - Actuadores}}
%\fontsize{14pt}{15}\selectfont
%\begin{minipage}[c]{1.0\linewidth}
%\begin{minipage}[c]{0.6\linewidth}
%      \centering
%	\begin{itemize}
%		\item Inyección de $O_2$/$CO_2$
%		\vspace{10px}
%		\item Iluminación
%		\vspace{10px}
%		\item Recambio de agua
%		\vspace{10px}
%		\item Calentadores
%		\vspace{10px}
%		\item Otros
%			\begin{itemize}
%				\item Dosificadores de alimento/nutrientes
%				\item Refrigeración
%				\item etc...
%			\end{itemize}
%	\end{itemize}
% \end{minipage}
%  \begin{minipage}[c]{0.35\linewidth}
%	\begin{figure}[H]
%	%	{\includegraphics[width=1\textwidth]{./imagenes/actuador_pump}\vspace{5px}}
%	%	{\includegraphics[width=1\textwidth]{./imagenes/actuador_heater}}	
%	\end{figure}	  	  	
%  \end{minipage}
%\end{minipage}
%\end{frame}
%
%\begin{frame}{\textbf{¿Sobre qué hace falta alertar? - Alarmas}}
%\fontsize{14pt}{15}\selectfont
%\begin{itemize}
%	\item Aviso de tarea periódica de mantenimiento
%	\vspace{10px}
%	\item Nivel de agua bajo/alto en tanques auxiliares
%	\vspace{10px}
%	\item Parámetro fuera de rango
%	\vspace{10px}
%	\item Falla de sensor
%	\vspace{10px}
%	\item Otros	
%\end{itemize}	
%\end{frame}
%
%\begin{frame}{\textbf{Soluciones existentes}}
%\fontsize{14pt}{15}\selectfont
%\vspace{-10px}
%	\begin{figure}
%	\begin{align*} 
%		{\includegraphics[width=.6\textwidth]{./imagenes/reefkeeper.JPG}}\\
%		\vspace{15px}
%		{\includegraphics[width=.6\textwidth]{./imagenes/arduino.jpg}}
%	\end{align*}
%	\end{figure}	  	  	
%\end{frame}
%%
%
%
%
%\section[Gestión]{Gestión de proyecto}
%\subsection[Interesados]{Identificación de los Interesados}
%
%\begin{frame}{\textbf{Identificación de los interesados}}
%\fontsize{12pt}{15}\selectfont
%\begin{itemize}
%		\item \textbf{Cliente:} Dr. Ing. Ariel Lutemberg, director de la Carrera de Esp. en Sistemas Embebidos.
%		\vspace{2px}
%		\item \textbf{Sponsors:} LSE, Facultad de Ingeniería, UBA.
%		\vspace{2px}
%		\item \textbf{End Users:} Particulares / Tiendas / Productores.
%		\vspace{2px}
%		\item \textbf{Champion:} Dr. Ing. Ariel Lutemberg, Coordinador Proyecto CIAA.
%		\vspace{2px}
%		\item \textbf{Drivers:} Carolina González, Esteban de Mateo.
%		\vspace{2px}
%		\item \textbf{Supporters} Ing. Alejandro Celery
%		\vspace{2px}
%		\item \textbf{Project Manager:} Ing. Patricio Bos
%		\vspace{2px}
%		\item \textbf{Team Members:} Ing. Nicolás Moretti
%	\end{itemize}
%\end{frame}
%
%\subsection[Scope]{Project Scope Statement}
%
%\begin{frame}{\textbf{Project Scope Statement}}
%\fontsize{14pt}{15}\selectfont
%\begin{itemize}
%\item \textbf{Objetivo:} Desarrollar un software que permita usar la CIAA como controlador de acuario.
%\vspace{5px}
%\item \textbf{Alcance:} Se limitan la cantidad de sensores, actuadores y alarmas.
%\vspace{5px}
%\item \textbf{Criterio de aceptación:} Visualizar en una interfaz web el estado del sistema y poder operar el conjunto de actuadores desde la misma interfaz.
%\vspace{5px}
%\item \textbf{Restricciones:} Finalizar el projecto antes del 15 de Diciembre de 2015.
%\end{itemize}
%\end{frame}
%
%\subsection[WBS]{Work Breakdown Structure}
%
%\begin{frame}{\textbf{WBS} Desglose de tareas}
%\fontsize{14pt}{15}\selectfont
%	\begin{enumerate}
%		\item Gestión del projecto
%		\vspace{10px}
%		\item Hardware
%		\vspace{10px}
%		\item Firmware
%		\vspace{10px}
%		\item Interfaz web
%		\vspace{10px}
%		\item Verificación y Validación
%		\vspace{10px}
%	\end{enumerate}
%\end{frame}


%\begin{frame}{\textbf{Activity on Node}}
%\fontsize{14pt}{15}\selectfont
%	\begin{figure}[H]
%	  	\includegraphics[width=1\textwidth]{./imagenes/AoN.pdf}
%	\end{figure}	  
%\end{frame}
%
%\subsection[Calidad]{Gestión de Calidad}
%
%\begin{frame}{\textbf{Gestión de Calidad}}
%\fontsize{14pt}{15}\selectfont
%	\begin{figure}[H]
%	  	\includegraphics[width=1\textwidth]{./imagenes/tabla-calidad.pdf}
%	\end{figure}	  
%\end{frame}
%
%\subsection[Recursos]{Gestión de Recursos}
%
%\begin{frame}{\textbf{Gestión de Recursos y Costos}}
%\fontsize{14pt}{15}\selectfont
%	\begin{figure}[H]
%	  	\includegraphics[width=1\textwidth]{./imagenes/tabla-recursos.pdf}\vspace{10px}
%	  	\includegraphics[width=.6\textwidth]{./imagenes/tabla-presupuesto.pdf}
%	\end{figure}	  
%
%\end{frame}
%
%\subsection[Riesgos]{Gestión de Riesgos}
%
%\begin{frame}{\textbf{Gestión de Riesgos}}
%\fontsize{14pt}{15}\selectfont
%	\begin{figure}[H]
%	  	\includegraphics[width=1\textwidth]{./imagenes/tabla-riesgo.pdf}
%	\end{figure}	  
%\end{frame}


\begingroup
\makeatletter
\setlength{\hoffset}{-.5\beamer@sidebarwidth}
\makeatother
\begin{frame}[plain,noframenumbering]
\fontsize{18pt}{15}\selectfont
\begin{center}
	MUCHAS GRACIAS POR SU ATENCIÓN\\
	\vspace{2cm}
	¿PREGUNTAS?
	\begin{minipage}[c]{0.35\linewidth}
		\begin{figure}[H]
			\includegraphics[width=1\textwidth]{./imagenes/red.jpg}
		\end{figure}	  	  	
	\end{minipage}
\end{center}
\end{frame}
\endgroup

\end{document}